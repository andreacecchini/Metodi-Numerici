\documentclass{article}
\usepackage{alltt, fancyvrb, url}
\usepackage{graphicx}
\usepackage[utf8]{inputenc}
\usepackage{float}
\usepackage{hyperref}
\usepackage{amsthm}
\newtheorem{definition}{Definition}
% Questo commentalo se vuoi scrivere in inglese.
\usepackage[italian]{babel}
\usepackage[italian]{cleveref}
% Intestazione degli appunti
\author{Andrea Cecchini}
\title{\textbf{Metodi Numerici per \\ L'intelligenza Artificiale.}}
\date{\today}
% Inizio del documento
\begin{document}
\maketitle
\newpage
\tableofcontents
\newpage
% Section: Introduzione all'Analisi Numerica.
\section{Introduzione all'Analisi Numerica.}
\subsection{Analisi Numerica.}
Introduciamo nel definire il compito dell'analisi numerica.
%
\begin{definition}[\textbf{Analisi Numerica}] L'Analisi Numerica e` la parte di matematica 
%
    che si occupa di dare una \textbf{risposta numerica} ad un problema matematico
%
    che modellizza un problema reale. 
\end{definition}
\subsubsection{Fasi della risoluzione di un problema numerico.}
%
Al fine di raggiungere tale problema, ci avvaliamo delle seguenti fasi:
%
\begin{itemize}
    \item \textbf{Tradurre} il problema reale in un insieme di equazioni 
%
    matematiche in grado di descriverlo
    \item \textbf{Trasformare} il problema matematica nel continuo in un 
%
    problema numerico discreto che sia risolubile.
    \item \textbf{Trasportare}  il problema discreto in un calcolatore mediante 
%
    l’applicazione di algoritmi numerici capaci di determinare la soluzione 
%
    in un tempo ottimale.
    \item \textbf{Interpretare e verificare} la soluzione numerica nei termini 
%
    della situazione reale e verificare così sia l’adeguatezza del modello 
%
    matematico sia l’efficienza dell’algoritmo risolutivo.
\end{itemize}
\subsubsection{Errori nel risolvere un problema numerico.}
Nel percorso appena descritto vi possono essere numerevoli errori, 
%
le quali sorgenti sono:
\begin{itemize}
    \item \textbf{Errori nel modello matematico} Nascono da una cattiva 
%
    traduzione del problema reale a quello matematico, per esempio si 
%
    considerano alcune cose come trascurabili quando non lo sono.
    \item \textbf{Errori nel modello numerico-computazionale} Vengono 
%
    descritti come errori di \textit{discretizzazione} o \textit{troncamento}.
    \item \textbf{Errori presenti nei dati} Nati da uno strumento di
%
    misurazione fallace o da misurazioni che possono essere influenzate
%
    da errori sistematici.
    \item \textbf{Errori di arrotondamento nei dati e nei calcoli} Sono 
%
    gli errori introdotti nella rappresentazone dei numeri sul calcolatore.
\end{itemize}
\newpage
\subsection{Classificazione dei problemi numerici}
\subsubsection{Problema numerico}
\begin{definition}[\textbf{Problema Numerico}] Per \textbf{problema numerico}
%
    intendiamo una descrizione chiara di una \textbf{relazione funzionale}  
%
    tra i dati (\textbf{input}) e i risultati (\textbf{output}).
\end{definition}
In particolare, in un problema numerico abbiamo i seguenti elementi:
\begin{itemize}
    \item \textbf{F} rappresenta la relazione funzionale tra input ed output.
    \item \textbf{x} rappresenta l’output della funzione di un determinato
%
    input.
    \item \textbf{y} rappresenta il dato di input della relazione funzionale.
\end{itemize}
\subsubsection{Classificazione dei problemi numerici.} Descritti questi 
%
3 elementi, è possibile classificare il problema numerico in base a 
%
cosa stiamo cercando:
\begin{itemize}
    \item \textbf{Problema diretto} \textbf{F} e \textbf{x} sono dati,
%
    bisogna \textbf{trovare y}.
    \item \textbf{Problema inverso} \textbf{F} e \textbf{y} sono dati,
%
    bisogna \textbf{trovare}.
    \item \textbf{Problema di identificazione} \textbf{x} e \textbf{y} sono noti,
%
    bisogna trovare \textbf{F}.
\end{itemize}
Quest’ultimo problema è quello che interesserà di più durante il corso, 
%
perchè è proprio il problema numerico che l’intelligenza artificiale
%
cerca di risolvere.

% End Section: Introduzione all'Analisi Numerica.
\end{document}

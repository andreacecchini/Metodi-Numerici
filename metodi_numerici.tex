\documentclass{article} 
\usepackage{tcolorbox}
\usepackage{alltt, fancyvrb, url}
\usepackage{graphicx}
\usepackage[utf8]{inputenc}
\usepackage{float}
\usepackage{hyperref}
\usepackage{amsthm}
\usepackage{amsmath}
\newtheorem{definition}{Definition}
% Questo commentalo se vuoi scrivere in inglese.
\usepackage[italian]{babel}
\usepackage[italian]{cleveref}

\tcbuselibrary{skins}

\usepackage{background}
\SetBgHshift{-2.3cm}
\SetBgVshift{0cm}

\usepackage[margin=3cm]{geometry}

\usepackage{tikz}
\usepackage{tikzpagenodes}


\usepackage{xparse}
\DeclareDocumentCommand\topic{ m m g g g g g}
{
\begin{tcolorbox}[sidebyside,sidebyside align=top,opacityframe=0,opacityback=0,opacitybacktitle=0, opacitytext=1,lefthand width=.3\textwidth]
\begin{tcolorbox}[colback=red!05,colframe=red!25,sidebyside align=top,width=\textwidth,before skip=0pt]
#1.\end{tcolorbox}%
\tcblower
\begin{tcolorbox}[colback=blue!05,colframe=blue!10,width=\textwidth,before skip=0pt]
#2
\end{tcolorbox}
\IfNoValueF {#3}{
\begin{tcolorbox}[colback=blue!05,colframe=blue!10,width=\textwidth]
#3
\end{tcolorbox}
}
\IfNoValueF {#4}{
\begin{tcolorbox}[colback=blue!05,colframe=blue!10,width=\textwidth]
#4
\end{tcolorbox}
}
\IfNoValueF {#5}{
\begin{tcolorbox}[colback=blue!05,colframe=blue!10,width=\textwidth]
#5
\end{tcolorbox}
}
\IfNoValueF {#6}{
\begin{tcolorbox}[colback=blue!05,colframe=blue!10,width=\textwidth]
#6
\end{tcolorbox}
}
\IfNoValueF {#7}{
\begin{tcolorbox}[colback=blue!05,colframe=blue!10,width=\textwidth]
#7
\end{tcolorbox}
}
\end{tcolorbox}
}

\def\summary#1{
\begin{tikzpicture}[overlay,remember picture,inner sep=0pt, outer sep=0pt]
\node[anchor=south,yshift=-1ex] at (current page text area.south) {% 
\begin{minipage}{\textwidth}%%%%
\begin{tcolorbox}[colframe=white,opacityback=0]
\begin{tcolorbox}[enhanced,colframe=black,fonttitle=\large\bfseries\sffamily,sidebyside=true, nobeforeafter,before=\vfil,after=\vfil,colupper=black,sidebyside align=top, lefthand width=.95\textwidth,opacitybacktitle=1, opacitytext=1,
segmentation style={black!55,solid,opacity=0,line width=3pt},
title=Summary
]
#1
\end{tcolorbox}
\end{tcolorbox}
\end{minipage}
};
\end{tikzpicture}
}

\usepackage{listings}
\usepackage{color}

\definecolor{dkgreen}{rgb}{0,0.6,0}
\definecolor{gray}{rgb}{0.5,0.5,0.5}
\definecolor{mauve}{rgb}{0.58,0,0.82}

\lstset{frame=tb,
  language=Python,
  aboveskip=3mm,
  belowskip=3mm,
  showstringspaces=false,
  columns=flexible,
  basicstyle={\small\ttfamily},
  numbers=none,
  numberstyle=\tiny\color{gray},
  keywordstyle=\color{blue},
  commentstyle=\color{dkgreen},
  stringstyle=\color{mauve},
  breaklines=true,
  breakatwhitespace=true,
  tabsize=3,
  numbers=left
}

% Intestazione degli appunti
\author{Andrea Cecchini}
\title{\textbf{Metodi Numerici per \\ L'intelligenza Artificiale.}}
\date{\today}
% Inizio del documento
\begin{document}
\maketitle
\newpage
\tableofcontents
\newpage
% Section: Introduzione all'Analisi Numerica.
\section{Introduzione all'Analisi Numerica.}
\subsection{Analisi Numerica.}
Introduciamo nel definire il compito dell'analisi numerica.
%
\topic{\textbf{Analisi Numerica}}
{
    L'Analisi Numerica è la parte di matematica 
    %
    che si occupa di dare una \textbf{risposta numerica} 
    %
    ad un problema matematico  che modellizza un problema reale.    
}
\subsubsection{Fasi della risoluzione di un problema numerico.}
%
Al fine di raggiungere tale problema, ci avvaliamo delle seguenti fasi:
%
\begin{itemize}
    \item \textbf{Tradurre} il problema reale in un insieme di equazioni 
%
    matematiche in grado di descriverlo
    \item \textbf{Trasformare} il problema matematica nel continuo in un 
%
    problema numerico discreto che sia risolubile.
    \item \textbf{Trasportare}  il problema discreto in un calcolatore mediante 
%
    l’applicazione di algoritmi numerici capaci di determinare la soluzione 
%
    in un tempo ottimale.
    \item \textbf{Interpretare} la soluzione numerica nei termini 
%
    della situazione reale e \textbf{verificare} così sia l’adeguatezza del modello 
%
    matematico sia l’efficienza dell’algoritmo risolutivo.
\end{itemize}
\subsubsection{Errori nel risolvere un problema numerico.}
Nel percorso appena descritto vi possono essere numerevoli errori, 
%
le quali sorgenti sono:
\begin{itemize}
    \item \textbf{Errori nel modello matematico} Nascono da una cattiva 
%
    traduzione del problema reale a quello matematico, per esempio si 
%
    considerano alcune cose come trascurabili quando non lo sono.
    \item \textbf{Errori nel modello numerico-computazionale} Vengono 
%
    descritti come errori di \textit{discretizzazione} o \textit{troncamento}.
    \item \textbf{Errori presenti nei dati} Nati da uno strumento di
%
    misurazione fallace o da misurazioni che possono essere influenzate
%
    da errori sistematici.
    \item \textbf{Errori di arrotondamento nei dati e nei calcoli} Sono 
%
    gli errori introdotti nella rappresentazone dei numeri sul calcolatore.
\end{itemize}
\newpage
\subsection{Classificazione dei problemi numerici}
\subsubsection{Problema numerico}
\begin{definition}[\textbf{Problema Numerico}] Per \textbf{problema numerico}
%
    intendiamo una descrizione chiara di una \textbf{relazione funzionale}  
%
    tra i dati (\textbf{input}) e i risultati (\textbf{output}).
\end{definition}
In particolare, in un problema numerico abbiamo i seguenti elementi:
\begin{itemize}
    \item \textbf{F} rappresenta la relazione funzionale tra input ed output.
    \item \textbf{x} rappresenta il dato di input della relazione funzionale.
    \item \textbf{y} rappresenta l’output della funzione di un determinato input
\end{itemize}
\subsubsection{Classificazione dei problemi numerici.} Descritti questi 
%
3 elementi, è possibile classificare il problema numerico in base a 
%
cosa stiamo cercando:
\begin{itemize}
    \item \textbf{Problema diretto} \textbf{F} e \textbf{x} sono dati,
%
    bisogna \textbf{trovare y}.
    \item \textbf{Problema inverso} \textbf{F} e \textbf{y} sono dati,
%
    bisogna \textbf{trovare x}.
    \item \textbf{Problema di identificazione} \textbf{x} e \textbf{y} sono noti,
%
    bisogna trovare \textbf{F}.
\end{itemize}
Quest’ultimo problema è quello che interesserà di più durante il corso, 
%
perchè è proprio il problema numerico che l’intelligenza artificiale
%
cerca di risolvere.

\summary{
    Abbiamo introdotto la materia dell'\textbf{analisi numerica}
%
    e quello che si prefissa di risolvere.
%
    Successivamente abbiamo definito il concetto di 
%    
    \textbf{problema numerico} ed abbiamo elencato i diversi tipi,
%
    quali \textbf{problema diretto}, \textbf{problema inverso} e
%
    \textbf{problema di identificazione}.
}
% End Section: Introduzione all'Analisi Numerica.
\newpage
\section{Introduzione all'Intelligenza artificiale}

\end{document}
